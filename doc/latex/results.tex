\chapter{Results and Discussion}\label{cp:res}

The goal of this project was to be able to conduct experiments with test subjects and to record datasets. Unfortunately, due to the ongoing situation when this project was conducted, this was not possible. Even though all possible safety measurements were considered and approved by the University's risk assessment department, the approval process was rendered impossible to go through within the given time frame. Apart from not being able to record any datasets, the algorithm might suffer from systematic errors due to subjective charasteristics in the developers blood pressure oscillations. 


This chapter highlights important results of the project.


- valve needs to open continuously to have a steady stream of 3mmHg/s
  opening introduces noise

- many algorithms only work for clean data

- MAP is used for diagnostics 

- why are microphones not used to determine blood pressure with the karakoff sounds?


\subsection{General Problems with Algorithms}
most algorithms expect perfect data to be 'perfect'

natural variablility of blood pressure during measurement
left and right arm difference 10mmHg normal


mmHg is no longer a SI unit since 2019