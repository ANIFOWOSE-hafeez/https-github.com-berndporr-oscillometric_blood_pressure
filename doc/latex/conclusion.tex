\chapter{Conclusion}\label{cp:concl} %TODO probably redundant to results?

This chapter discusses the outcome of the project and proposes the next steps to take.

The goal of this project was to be able to conduct experiments with test subjects and to record datasets. Unfortunately, due to the ongoing situation when this project was done, this was not possible. Even though all possible safety measurements were considered and approved by the University's risk assessment department, the approval process was rendered impossible to go through within the given time frame. Apart from not being able to record any datasets, the algorithm might suffer from systematic errors due to subjective characteristics in the developer's blood pressure oscillations. 


\section{Application and Algorithm Evaluation}

The developed program guides the user through taking their measurements. This application could be used in the future to acquire datasets. The prefered way is that the test subject does not move. This can be achieved if an examiner performs inflation and deflation of the cuff. The instructions in the GUI are designed to obey social distancing guidelines and would have to be updated. 

An even better option is equipping the setup with an electric pump and valve, so no human interaction is required to take the measurements. It would also make the process more stable. 

A common problem during deflation is inconsistency. For the implemented algorithm, a consistent deflation rate is more important than the exact speed of 3 mmHg/s. Therefore, the decision was made not to pursue the implementation of a more detailed deflation rate feedback. It would only encourage the user to change the valve opening more frequently to get the perfect deflation speed, which is not needed. The best feedback for consistency is a steadily falling needle. Excessively fast or slow deflation can cause inaccuracies in the measurement, but the algorithm will still determine results. The OMWE will have fewer or more values. Fewer values can mean not enough data to assess BP accurately. Too many can result in a flat OMWE with no clear maxima.

If the deflation is too fast and the user closes the valve, this influences oscillations. A large amplitude might be observed where a small one is expected. Similarly, if deflation is continuous and at a correct speed of 3 mmHg/s in high-pressure regions, it is possible to get stuck around MAP. The deflation rate decreases if the opening of the valve stays the same as the pressure in the cuff decreases. Often, this is minor, but if the starting pressure is considerably above BP, it can cause the deflation to stop around MAP. In this case, the valve has to be opened further. If this is done carelessly, deflation will be too fast. An automated valve would solve this problem. 

A second problem is if the user is swift to take the measurement, pumps up the pressure quickly and immediately releases it at the right speed, oscillations from the quick changes before deflation started can have an impact on the measurement. They might have a valid pulse peak but an invalidly large amplitude. Worst case scenario, this peak will be detected as MAP. Because the OMWE is calculated after data acquisition, changing this part of the algorithm is not trivial. The first 1.2 s of the deflation is ignored to account for this. Increasing this value could solve this problem, but might introduce new ones if it is too high. 

As stated before, the MAP is the only characteristic that can be determined confidently from the oscillations. The implemented GUI tries to address this by displaying the results in a fashion that makes it clear that only the MAP should be trusted. Because many people know about systolic and diastolic BP, but not about the MAP, they are still included. The ratios used to calculate SBP and DBP can easily be adapted in the settings menu.


\section{Proposed Software Improvements}

Because no experiments could be performed with the implemented application, the algorithm may have a subjective bias. Hopefully, this can be improved by adjusting the configurable parameters in the OBPDetection class. In any way, a series of tests, using people of different age and health should be conducted next. 

There is currently no implementation of deflation detection. The algorithm assumes a continuously deflating pressure curve but does not check for it. If an electric pump is available and controllable in software, it would be possible to measure BP during inflation of the cuff. Provided the pump does not add further noise. At the very least, the linearly increasing pressure could be used to estimate the range of BP and adjust the maximally pumped-up value accordingly.

As every other algorithm that was discussed before, the implemented one is susceptible to noise and artefacts, e.g. movements. From the information gathered in chapter \ref{cp:theory}, this is due to the method of oscillometry in general. However, some aspects have room for improvements. The maxima and minima detection are dependent and rely on a minimum height for the maxima. A possibility is the Persistence1D class by \citet{Weinkauf2020}. It is written for precisely this purpose, but was found late in the project and was not investigated.

Furthermore, smoothing the calculated OMWE has the potential to improve repetition accuracy because small variations would have less of an impact.

\section{Beyond Oscillometry}

Blood pressure used to be measured with the auscultatory method, which is suitable to determine SBP and DBP accurately. Oscillometric blood pressure measurements are not suited for that, as the last 40 years of research has shown. However, it is useful to determine the MAP, and it should be used for that. A YouTube video by \citet{Joe2019} addressed at nursed working in the intensive care unit (ICU), who calculate MAP from the SBP and DBP values, displays how little medical professionals are aware of this. Manufacturers of automatic devices should avoid giving inaccurate estimates of systolic and diastolic BP or make it clear that MAP is more precise.

If systolic and diastolic BP is needed, other methods should be considered. Using a microphone to listen to Korotkoff sounds while the cuff is deflation is an inexpensive addition to the system that renders ratios useless and improves accuracy for those values. The test done in \ref{sec:koro} shows this. Automated devices that use this approach exist, but they rare.

Another method is to use pulse transit time (PTT) to improve the accuracy of measurements, as proposed by \citet{Forouzanfar2015}. 

It is essential to know that blood pressure varies naturally to a great extent, also during a measurement. Additionally, it depends on which arm it was performed on, the difference between them can be up to 10 mmHg at any given time. This limits the accuracy of blood pressure measurements in general. After all, it is a physiological parameter and not a fixed value.

%Moreover, mmHg is no longer a unit in the SI brochure since 2019, because it is dependant on where it is measured. However, it is hard to imagine the medical field to stop using it. 
